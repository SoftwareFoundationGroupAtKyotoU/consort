% This is samplepaper.tex, a sample chapter demonstrating the
% LLNCS macro package for Springer Computer Science proceedings;
% Version 2.20 of 2017/10/04
%
\documentclass[runningheads]{llncs}
%
\usepackage{bcprules}
\usepackage{graphicx}
% Used for displaying a sample figure. If possible, figure files should
% be included in EPS format.
%
% If you use the hyperref package, please uncomment the following line
% to display URLs in blue roman font according to Springer's eBook style:
% \renewcommand\UrlFont{\color{blue}\rmfamily}

\usepackage{amsmath}
\usepackage{amssymb}

\DeclareMathOperator{\SEQ}{;}
\newcommand\expr{e}
\newcommand\intconst{n}
\newcommand\SKIP{\mathbf{skip}}
\DeclareMathOperator{\BOP}{\mathit{bop}}
\DeclareMathOperator{\LET}{\mathbf{let}}
\DeclareMathOperator{\IN}{\mathbf{in}}
\DeclareMathOperator{\WRITE}{:=}
\newcommand{\DEREF}{*}
\newcommand{\NULL}{\mathbf{null}}
\DeclareMathOperator\IFZERO{\mathbf{ifz}}
\DeclareMathOperator\THEN{\mathbf{then}}
\DeclareMathOperator\ELSE{\mathbf{else}}
\DeclareMathOperator{\COL}{:}
\newcommand\ASSERT{\mathbf{assert}}

\newcommand\decl{d}
\newcommand\prog{P}

\newcommand\tuple[1]{\left\langle{#1}\right\rangle}

\newcommand\stmt{s}

\newcommand\typ{\tau}

\newcommand\TINT{\mathbf{int}}
\DeclareMathOperator\TREF{\mathbf{ref}}

\newcommand\ownership{r}

\newcommand\RAT{\mathbb{Q}}

\newcommand\set[1]{\left\{{#1}\right\}}


\begin{document}
%
\title{Context- and Flow-Sensitive Refinement Types for Imperative Programs}
%
%\titlerunning{Abbreviated paper title}
% If the paper title is too long for the running head, you can set
% an abbreviated paper title here
%
\author{Ren Siqi \and Kohei Suenaga \and Atsushi Igarashi \and Naoki Kobayashi}
%
\authorrunning{Ren Siqi et al.}
% First names are abbreviated in the running head.
% If there are more than two authors, 'et al.' is used.
%
\institute{}
%
\maketitle              % typeset the header of the contribution
%
\begin{abstract}

\keywords{}
\end{abstract}
%
%
%

\section{Simple setting without indirection of pointers and without subtyping}

\subsection{Language}

\[
\begin{array}{rcl}  
    \decl &::=& \set{f \mapsto (x_1,\dots,x_n)\expr}\\
    % \expr &::=& x \mid \DEREF x \mid \intconst \mid \NULL\\
    \expr &::=& \LET x = \MKREF y \IN \expr \mid \LET x = y \IN \expr \mid \LET x = \intconst \IN \expr \mid \LET x = f^l(e_1,\dots,e_n) \IN \expr\\
    &\mid& \LET x = \NULL \IN \expr \mid \IFZERO x \THEN \expr_1 \ELSE \expr_2 \mid \ASSERT(\expr_1 = \expr_2) \mid \intconst \mid x \mid x \WRITE y \SEQ \expr\\
    \prog &::=& \tuple{\decl_1 \cup \dots \cup \decl_n, \expr}\\
    \typ &::=& \set{x \COL \TINT \mid \varphi} \mid \set{x \COL \TINT \mid \varphi} \TREF^{\ownership}\\
    \funtyp &::=& 
    \ownership &\in& [0,1] \subseteq \RAT
\end{array}
\]

\subsection{Type system}

\paragraph{Type judgment.}

\begin{figure}[t]

\infrule[T-Call]
{}
{\funenv; [l\beta/\alpha](x_1\COL\typ_1,\dots,x_n\COL\typ_n); \beta \p \LET y = f^l(x_1,\dots,x_n) \IN e \COL \typ \produces \tenv\diff y}

\label{Typing rules.}
\label{fig:typingRules}
\end{figure}

\bibliographystyle{plain}
\bibliography{main.bib}

\end{document}

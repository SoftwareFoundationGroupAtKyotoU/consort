% This is samplepaper.tex, a sample chapter demonstrating the
% LLNCS macro package for Springer Computer Science proceedings;
% Version 2.20 of 2017/10/04
%
\documentclass[runningheads]{llncs}
%
\usepackage{bcprules}
\usepackage{graphicx}
% Used for displaying a sample figure. If possible, figure files should
% be included in EPS format.
%
% If you use the hyperref package, please uncomment the following line
% to display URLs in blue roman font according to Springer's eBook style:
% \renewcommand\UrlFont{\color{blue}\rmfamily}

\usepackage{amsmath}
\usepackage{amssymb}

\DeclareMathOperator{\SEQ}{;}
\newcommand\expr{e}
\newcommand\intconst{n}
\newcommand\SKIP{\mathbf{skip}}
\DeclareMathOperator{\BOP}{\mathit{bop}}
\DeclareMathOperator{\LET}{\mathbf{let}}
\DeclareMathOperator{\IN}{\mathbf{in}}
\DeclareMathOperator{\WRITE}{:=}
\newcommand{\DEREF}{*}
\newcommand{\NULL}{\mathbf{null}}
\DeclareMathOperator\IFZERO{\mathbf{ifz}}
\DeclareMathOperator\THEN{\mathbf{then}}
\DeclareMathOperator\ELSE{\mathbf{else}}
\DeclareMathOperator{\COL}{:}
\newcommand\ASSERT{\mathbf{assert}}

\newcommand\decl{d}
\newcommand\prog{P}

\newcommand\tuple[1]{\left\langle{#1}\right\rangle}

\newcommand\stmt{s}

\newcommand\typ{\tau}

\newcommand\TINT{\mathbf{int}}
\DeclareMathOperator\TREF{\mathbf{ref}}

\newcommand\ownership{r}

\newcommand\RAT{\mathbb{Q}}

\newcommand\set[1]{\left\{{#1}\right\}}


\begin{document}
%
\title{Context- and Flow-Sensitive Refinement Types for Imperative Programs}
%
%\titlerunning{Abbreviated paper title}
% If the paper title is too long for the running head, you can set
% an abbreviated paper title here
%
\author{Ren Siqi \and Kohei Suenaga \and Atsushi Igarashi \and Naoki Kobayashi}
%
\authorrunning{Ren Siqi et al.}
% First names are abbreviated in the running head.
% If there are more than two authors, 'et al.' is used.
%
\institute{}
%
\maketitle              % typeset the header of the contribution
%
\begin{abstract}

\keywords{}
\end{abstract}
%
%
%

\section{Simple setting without indirection of pointers and without subtyping}

\subsection{Language}

\[
\begin{array}{rcl}  
  \decl &::=& \set{f \mapsto (x_1,\dots,x_n)\expr}\\
  % \expr &::=& x \mid \DEREF x \mid \intconst \mid \NULL\\
  \expr &::=& \LET x = \MKREF y \IN \expr \mid \LET x = y \IN \expr \mid \LET x = \intconst \IN \expr \mid \LET x = f^l(e_1,\dots,e_n) \IN \expr\\
        &\mid& \LET x = \NULL \IN \expr \mid \IFZERO x \THEN \expr_1 \ELSE \expr_2 \mid \ASSERT(\expr_1 = \expr_2) \mid \intconst \mid x \mid x \WRITE y \SEQ \expr\\
  \prog &::=& \tuple{\decl_1 \cup \dots \cup \decl_n, \expr}\\
  \typ &::=& \set{x \COL \TINT \mid \varphi} \mid \set{x \COL \TINT \mid \varphi} \TREF^{\ownership}\\
  \funtyp &::=& \tuple{x_1\COL\typ_1,\dots,x_n\COL\typ_n}\ra\tuple{x_1\COL\typ_1',\dots,x_n\COL\typ_n' \mid \typ}\\
  \ownership &\in& [0,1] \subseteq \RAT\\
\end{array}
\]

\subsection{Type system}

\paragraph{Operations on types.}

\begin{figure}[t]
  \infrule[Tadd-Ref1]
  {u_1 \ne 0 \andalso u_2 \ne 0}
  {\set{x \COL \TINT \mid \varphi} \TREF^{\ownership_1} + \set{x \COL \TINT \mid \varphi} \TREF^{\ownership_2} = \set{x \COL \TINT \mid \varphi} \TREF^{\ownership_1 + \ownership_2}}

  \infax[Tadd-Ref2]
  {\typ_1 \TREF^0 + \typ_2 \TREF^{\ownership} = \typ_2 \TREF^{\ownership}}

  \caption{Rules for $\typ_1 \addt \typ_2$.}
  \label{fig:addition}
\end{figure}

\begin{definition}
  $\typ_1 + \typ_2$ is the least commutative operation that satisfies the rules in Figure~\ref{fig:addition}.
\end{definition}

\paragraph{Subtyping.}

\begin{figure}[t]

  \infrule[Sub-Int1]
  {
    \tenv \models \varphi_1 \iff \varphi_2
  }
  {\tenv \vdash \set{x \COL \TINT \mid \varphi_1} \TREF^1 \subt \set{x \COL \TINT \mid \varphi_2} \TREF^1}

  \infrule[Sub-Int2]
  {
    0 < u < 1
    \andalso
    \tenv \models \varphi_1 \implies \varphi_2
  }
  {\tenv \vdash \set{x \COL \TINT \mid \varphi_1} \TREF^{u} \subt \set{x \COL \TINT \mid \varphi_2} \TREF^{u}}

  \infrule[Sub-Ref]
  {
    u_1 \ge u_2
    \andalso
    u_2 \ge 1 \implies \tenv \vdash \typ_2 \subt \typ_1
    \andalso
    u_2 > 0 \implies \tenv \vdash \typ_1 \subt \typ_2
  }
  {\tenv \vdash \typ_1 \TREF^{u_1} \subt \typ_2 \TREF^{u_2}}
  
  \caption{Rules for subtyping.}
  \label{fig:subtyping}
\end{figure}

\paragraph{Type judgment.}

\begin{figure}[t]

  \infrule[T-MkRef]
  {\funenv \mid \tenv, x \COL \typ_1 \TREF^1, y \COL \typ_2 \mid \beta \vdash \expr \COL \typ' \produces \tenv'
    \andalso
    x \notin \fv(\typ')}
  {\funenv \mid \tenv, y \COL (\typ_1 + \typ_2) \mid \beta \p \LET x = \MKREF y \IN \expr \COL \typ' \produces \tenv' \diff x}

  \rulesp
  
  \infrule[T-Call]
  {
    \funenv(f) = \forall \alpha. \tuple{x_1\COL\typ_1,\dots,x_n\COL\typ_n} \ra \tuple{x_1\COL\typ_1',\dots,x_n\COL\typ_n' \mid \typ}\\
    \funenv \mid [l\beta/\alpha](x_1\COL\typ_1',\dots,x_n\COL\typ_n',y\COL\typ) \mid \beta \vdash \expr \COL \typ' \produces \tenv'
    \andalso
    y \notin \fv(\typ)
  }
  {\funenv \mid [l\beta/\alpha](x_1\COL\typ_1,\dots,x_n\COL\typ_n) \mid \beta \p \LET y = f^l(x_1,\dots,x_n) \IN \expr \COL \typ \produces \tenv' \diff y}

  \rulesp
  
  \infrule[T-FunDef]
  {
    \funenv(f) = \forall \alpha. \tuple{x_1\COL\typ_1,\dots,x_n\COL\typ_n}\ra\tuple{x_1\COL\typ_1',\dots,x_n\COL\typ_n' \mid \typ}\\
    \funenv \mid x_1\COL\typ_1,\dots,x_n\COL\typ_n \mid \alpha \vdash \expr \COL \typ \produces x_1\COL\typ_1',\dots,x_n\COL\typ_n'\\
    \alpha \notin \fcv(\funenv)
  }
  {\funenv \vdash \set{f \mapsto (x_1,\dots,x_n)\expr}}
  
\caption{Typing rules.}
\label{fig:typingRules}
\end{figure}

\bibliographystyle{plain}
\bibliography{main.bib}

\end{document}

%%% Local Variables:
%%% mode: latex
%%% TeX-master: t
%%% End:

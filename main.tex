% This is samplepaper.tex, a sample chapter demonstrating the
% LLNCS macro package for Springer Computer Science proceedings;
% Version 2.20 of 2017/10/04
%
\documentclass[runningheads]{llncs}
%
\usepackage{bcprules}\typicallabel{T-Hoge}
\usepackage{graphicx,color}
% Used for displaying a sample figure. If possible, figure files should
% be included in EPS format.
%
% If you use the hyperref package, please uncomment the following line
% to display URLs in blue roman font according to Springer's eBook style:
% \renewcommand\UrlFont{\color{blue}\rmfamily}

\usepackage{amsmath}
\usepackage{amssymb}
\usepackage{stmaryrd}

\DeclareMathOperator{\SEQ}{;}
\newcommand\expr{e}
\newcommand\intconst{n}
\newcommand\SKIP{\mathbf{skip}}
\DeclareMathOperator{\BOP}{\mathit{bop}}
\DeclareMathOperator{\LET}{\mathbf{let}}
\DeclareMathOperator{\IN}{\mathbf{in}}
\DeclareMathOperator{\WRITE}{:=}
\newcommand{\DEREF}{*}
\newcommand{\NULL}{\mathbf{null}}
\DeclareMathOperator\IFZERO{\mathbf{ifz}}
\DeclareMathOperator\THEN{\mathbf{then}}
\DeclareMathOperator\ELSE{\mathbf{else}}
\DeclareMathOperator{\COL}{:}
\newcommand\ASSERT{\mathbf{assert}}

\newcommand\decl{d}
\newcommand\prog{P}

\newcommand\tuple[1]{\left\langle{#1}\right\rangle}

\newcommand\stmt{s}

\newcommand\typ{\tau}

\newcommand\TINT{\mathbf{int}}
\DeclareMathOperator\TREF{\mathbf{ref}}

\newcommand\ownership{r}

\newcommand\RAT{\mathbb{Q}}

\newcommand\set[1]{\left\{{#1}\right\}}

\newcommand{\AI}[1]{\textcolor{blue}{[AI: #1]\marginpar{$\longleftarrow$}}}

\begin{document}
%
\title{Context- and Flow-Sensitive Refinement Types for Imperative Programs}
%
%\titlerunning{Abbreviated paper title}
% If the paper title is too long for the running head, you can set
% an abbreviated paper title here
%
\author{Ren Siqi \and Kohei Suenaga \and Atsushi Igarashi \and Naoki Kobayashi}
%
\authorrunning{Ren Siqi et al.}
% First names are abbreviated in the running head.
% If there are more than two authors, 'et al.' is used.
%
\institute{}
%
\maketitle              % typeset the header of the contribution
%
\begin{abstract}

\keywords{}
\end{abstract}
%
%
%

\section{Simple setting without indirection of pointers and without subtyping}

\newcommand\val{v}

\subsection{Language}

\[
  \begin{array}{rcl}
  \decl &::=& f \mapsto (x_1,\dots,x_n)\expr \\
  % \expr &::=& x \mid \DEREF x \mid \intconst \mid \NULL\\
  \expr &::= &
              \val \mid
              x \mid
              \LET x = y \IN \expr \mid
              \LET x = \val \IN \expr \mid
              \IFZERO x \THEN \expr_1 \ELSE \expr_2  \\ &\mid&
              \LET x = \MKREF y \IN \expr \mid
              \LET x = *y \IN \expr \mid
              \LET x = f^\ell(x_1,\dots,x_n) \IN \expr \\ &\mid&
              x \WRITE y \SEQ \expr \mid
              \ALIAS(\expr_1 = \expr_2) \SEQ \expr_3 \mid
              \ASSERT(\varphi) \SEQ \expr \\
    \val &::=& a \mid n\\
    \prog &::=& \tuple{\set{\decl_1, \dots, \decl_n}, \expr}\\
    \pp &\in& \textbf{Labels}\\
  \pps &\in& \textbf{Labels}^*
  \end{array}
\]

\(\varphi\) stands for a logical formula over integers and paths $\pps$).
We assume a set of \emph{variables}, which is ranged over by $x,y,z,\dots$, and a set of \emph{locations}, which is ranged over by $a,b,c,\dots$.

\subsection{Type system}

\[
\begin{array}{rcl}
  \typ &::=& \TUNIT \mid \set{x \COL \TINT \mid \varphi} \mid \set{x \COL \TINT \mid \varphi} \TREF^{\ownership}\\
  \funtyp &::=& \tuple{x_1\COL\typ_1,\dots,x_n\COL\typ_n}\ra\tuple{x_1\COL\typ_1',\dots,x_n\COL\typ_n' \mid \typ}\\
  \ownership &\in& [0,1] \subseteq \RAT
\end{array}
\]
\begin{definition}
  $\sem{\typ}_y$ is defined as follows:
  \[
    \begin{array}{rcl}
      \sem{\TUNIT}_y &=& \TRUE\\
      \sem{\set{x \COL \TINT \mid \varphi}}_y &=& [y/x]\varphi\\
      \sem{\set{x \COL \TINT \mid \varphi} \TREF^{\ownership}}_y &=& \TRUE.
    \end{array}
  \]
\end{definition}

\begin{definition}
  $\sem{\Gamma}$ is defined as follows:
  \[
    \begin{array}{rcl}
      \sem{\emptyset} &=& \TRUE\\
      \sem{\Gamma, x\COL\typ} &=& \sem{\Gamma} \land \sem{\typ}_x \\
      \sem{\Gamma, \varphi} & = & \sem{\Gamma} \land \varphi.
    \end{array}
  \]
\end{definition}

\begin{definition}
  We write $\Gamma \models \varphi$ if $\models \sem{\Gamma} \implies \varphi$ holds.
\end{definition}

\paragraph{Operations on types.}

\begin{figure}[t]
  \leavevmode
  \infax[Tadd-Unit]{
    \TUNIT + \TUNIT = \TUNIT
  }
  \infax[Tadd-Int]{
    \set{x \COL \TINT \mid \varphi} + \set{x \COL \TINT \mid \varphi} = \set{x \COL \TINT \mid \varphi}
    }
  \infrule[Tadd-Ref1]{
    \ownership_1 \ne 0 \andalso
    \ownership_2 \ne 0
  }{
    \set{x \COL \TINT \mid \varphi} \TREF^{\ownership_1} + \set{x \COL \TINT \mid \varphi} \TREF^{\ownership_2} = \set{x \COL \TINT \mid \varphi} \TREF^{\ownership_1 + \ownership_2}
  }
  \infrule[Tadd-Ref2]{
    r > 0
    }{
    \typ_1 \TREF^0 + \typ_2 \TREF^{\ownership} = \typ_2 \TREF^{\ownership}
  }
  \infax[Tadd-Ref3]{
    \typ_1 \TREF^0 + \typ_2 \TREF^0 = \set{x \COL \TINT \mid \textbf{true}} \TREF^0
  }

  \caption{Rules for $\typ_1 \addt \typ_2$.}
  \label{fig:addition}
\end{figure}

\begin{definition}
  $\typ_1 + \typ_2$ is the least commutative partial operation that satisfies the rules in Figure~\ref{fig:addition}.
  \AI{Rules for unit and int types.}
\end{definition}

\paragraph{Subtyping.}

\begin{figure}[t]
\leavevmode
  % \infrule[Sub-Int1]{
  %   \tenv \models \varphi_1 \iff \varphi_2
  % }{
  %   \tenv \vdash \set{x \COL \TINT \mid \varphi_1} \TREF^1 \subt \set{x \COL \TINT \mid \varphi_2} \TREF^1
  % }
  \infax[Sub-Unit]{
    \tenv \vdash \TUNIT \subt \TUNIT
  }
  \infrule[Sub-Int]{
    \tenv \models \varphi_1 \implies \varphi_2
  }{
    \tenv \vdash \set{x \COL \TINT \mid \varphi_1} \subt \set{x \COL \TINT \mid \varphi_2}
  }
  \infrule[Sub-Ref]{
    \ownership_1 \ge \ownership_2
    \andalso
    \ownership_2 \ge 1 \implies \tenv \vdash \typ_2 \subt \typ_1
    \andalso
    \ownership_2 > 0 \implies \tenv \vdash \typ_1 \subt \typ_2
  }{
    \tenv \vdash \typ_1 \TREF^{\ownership_1} \subt \typ_2 \TREF^{\ownership_2}
  }
  \infrule[Sub-TyEnv]{
    \tenv \vdash \tenv(x) \subt \tenv'(x) \quad (x \in dom(\tenv')) \\
    \models \sem{\tenv} \implies \sem{\tenv'}
  }{
    \tenv \subt \tenv'
  }
  \caption{Rules for subtyping.}
  \label{fig:subtyping}
\end{figure}

\paragraph{Type judgment.}

\AI{\rn{T-Let} loses information that $x$ is equal to $y$ (unlike \rn{T-LetInt}).}

\begin{figure}[t]
  \leavevmode
  \infrule[T-Int]{
  }{
    \funenv \mid \tenv
            \mid \beta \vdash n \COL \set{x \COL \TINT \mid x = \intconst} \COL \tenv
  }
  \infrule[T-Var]{
  }{
    \funenv \mid \tenv_1, x\COL(\typ_1+\typ_2),\tenv_2 \mid \beta \vdash x \COL  \typ_1 \produces \tenv_1, x\COL \typ_2, \tenv_2
  }
  \infrule[T-Let]{
    \funenv \mid \tenv_1, y \COL \typ_2, \tenv_2, x \COL \typ_1, x =_{\typ_1} y \mid \beta \p \expr \COL \typ' \produces \tenv'
    \andalso
    x \notin \fv(\typ')
  }{
    \funenv \mid \tenv_1, y\COL(\typ_1+\typ_2), \tenv_2 \mid \beta \p \LET x = y \IN \expr \COL \typ' \produces \tenv' \diff x
  }
  \infrule[T-LetInt]{
    \funenv \mid \tenv, x\COL\set{x \COL \TINT \mid x = \intconst} \mid \beta \p \expr \COL \typ' \produces \tenv'
    \andalso
    x \notin \fv(\typ')
  }{
    \funenv \mid \tenv \mid \beta \p \LET x = \intconst \IN \expr \COL \typ' \produces \tenv' \diff x
  }
  \infrule[T-If]{
    \tenv = \tenv_1, x : \set{y:\TINT \mid \varphi}, \tenv_2 \\
    \funenv \mid \tenv_1, x : \set{y:\TINT \mid \varphi \land y = 0}, \tenv_2 \mid \beta \vdash \expr_1 \COL \typ \produces \tenv' \\
    \funenv \mid \tenv_1, x : \set{y:\TINT \mid \varphi \land y \neq 0}, \tenv_2 \mid \beta \vdash \expr_2 \COL \typ \produces \tenv'
  }{
    \funenv \mid \tenv_1, x : \set{y:\TINT \mid \varphi}, \tenv_2 \mid \beta \vdash \IFZERO x \THEN \expr_1 \ELSE \expr_2 \COL \typ \produces \tenv'
  }
  \infrule[T-MkRef]{
    \funenv \mid \tenv_1, y \COL \typ_2, \tenv_2, x \COL \typ_1 \TREF^1 \mid \beta \vdash \expr \COL \typ' \produces \tenv'
    \andalso
    x \notin \fv(\typ')
  }{
    \funenv \mid \tenv_1, y \COL (\typ_1 + \typ_2), \tenv_2 \mid \beta \p \LET x = \MKREF y \IN \expr \COL \typ' \produces \tenv' \diff x
  }
  \infrule[T-Deref]{
    r > 0 \andalso
    \funenv \mid \tenv_1, y \COL \typ_1 \TREF^r, \tenv_2, x \COL \typ_2 \mid \beta \vdash \expr \COL \typ' \produces \tenv' \andalso
    x \not \in \fv(\typ')
  }{
    \funenv \mid \tenv_1, y \COL (\typ_1 + \typ_2) \TREF^r, \tenv_2 \mid \beta \vdash \LET x = *y \IN \expr \COL \typ' \produces \tenv' \diff x
  }
  \infrule[T-Call]{
    \funenv(f) = \forall \pps_1. \tuple{x_1\COL\typ_1,\dots,x_n\COL\typ_n} \ra \tuple{x_1\COL\typ_1',\dots,x_n\COL\typ_n' \mid \typ}\\
    \funenv \mid \tenv,[\pp\pps_2/\pps_1](x_1\COL\typ_1',\dots,x_n\COL\typ_n',y\COL\typ) \mid \pps_2 \vdash \expr \COL \typ' \produces \tenv'
    \andalso
    y \notin \fv(\typ)
  }{
    \funenv \mid \tenv,[\pp\pps_2/\pps_1](x_1\COL\typ_1,\dots,x_n\COL\typ_n) \mid \pps_2 \p \LET y = f^\pp(x_1,\dots,x_n) \IN \expr \COL \typ' \produces \tenv' \diff y
  }
  \infrule[T-Assign]{
    (\text{The shapes of $\typ'$ and $\typ_2$ are similar}) \\
    \funenv \mid \tenv_1,  y : \typ_2, \tenv_2, x : \typ_1 \TREF^1 \mid \beta \vdash  \expr \COL \typ \produces \tenv'
  }{
    \funenv \mid \tenv_1, y : (\typ_1+\typ_2), \tenv_2, x : \typ' \TREF^1 \mid \beta \vdash  x \WRITE y \SEQ \expr \COL \typ \produces \tenv'
  }
  \infrule[T-Alias]{
    \typ_1 \TREF^{r_1} + \typ_2 \TREF^{r_2} = \typ'_1 \TREF^{r'_1} + \typ'_2 \TREF^{r'_2}
    \\
    \funenv \mid \tenv, x : \typ'_1 \TREF^{r'_1}, y : \typ'_2 \TREF^{r'_2} \mid \beta \vdash \expr_3 \COL \tau \produces \tenv'
  }{
    \funenv \mid \tenv, x : \typ_1 \TREF^{r_1}, y : \typ_2 \TREF^{r_2} \mid \beta \vdash \ALIAS(x = y) \SEQ \expr_3 \COL \tau \produces \tenv'
  }
  \infrule[T-Assert]{
    \tenv \models \varphi \andalso
    \funenv \mid \tenv \mid \beta \vdash \expr \COL \tau \produces \tenv'
  }{
    \funenv \mid \tenv \mid \beta \vdash \ASSERT(\varphi)\SEQ \expr \COL \tau \produces \tenv'
  }
  \infrule[T-Sub]{
    \tenv \subt \tenv' \andalso
    \funenv \mid \tenv' \mid \beta \vdash \expr \COL \tau \produces \tenv'' \andalso
    \tenv'', x:\typ \subt \tenv''', x:\typ'
  }{
    \funenv \mid \tenv \mid \beta \vdash \expr \COL \typ' \produces \tenv'''
  }
  \infrule[T-FunDef]{
    \funenv(f) = \forall \alpha. \tuple{x_1\COL\typ_1,\dots,x_n\COL\typ_n}\ra\tuple{x_1\COL\typ_1',\dots,x_n\COL\typ_n' \mid \typ}\\
    \funenv \mid x_1\COL\typ_1,\dots,x_n\COL\typ_n \mid \alpha \vdash \expr \COL \typ \produces x_1\COL\typ_1',\dots,x_n\COL\typ_n'\\
    \alpha \notin \fcv(\funenv)
  }{
    \funenv \vdash f \mapsto (x_1,\dots,x_n)\expr
  }
  \infrule[T-Prog]{
    \funenv \vdash d_1 \quad \cdots \quad
    \funenv \vdash d_n \andalso
    \funenv \mid \emptyset \mid \varepsilon \vdash e : \TUNIT \produces \tenv'
  }{
    \vdash \tuple{\set{\decl_1, \dots, \decl_n}, \expr}
  }
\caption{Typing rules.}
\label{fig:typingRules}
\end{figure}

\AI{\rn{T-Let} discards all the refinement on \(x\).  Probably we should propagate it to $\tenv'$.}

\AI{\rn{T-Let} and \rn{T-LetInt} can be merged.}

\AI{Define \(x =_\typ y\).}

\AI{\rn{T-Call} requires actual arguments to be variables introduced at the end of the type environment.  This is not a restriction because one can introduce let-expressions before a function call and alias-expressions after the call.}

\AI{$\varepsilon$ in \rn{T-Prog} stands for the empty string.}

\AI{$\tenv_1, x:\typ \TREF^1, \tenv_2 \subt \tenv_1, \tenv_2, x:\typ \TREF^1$ should always hold ($\tenv_2$ cannot reference $x$).}

%edit by shiki
\subsection{Transition rules}
For each language expression occurs, we accomplish writing the rules for transition. Here, $\langle H, R,\overrightarrow{E}, E \rangle$ represent heap(memory),
register(environment), a sequence of context, program respectively.
It's obvious when $e$ is an integer $n$, when it's a variable $x$, in order to better handle with recursive types,
we refresh every variable $x$ to $x'$ after executing expression.
Define a named local expression $\LET x=y \IN e$, which $x$ can then be used later on in the function instead of $y$, so that the
register of $x$ update to the $R(y)$.
The expression $x:=y$ only update the register of $x$ to $y$.
The same changes happen when $\LET x=n \IN e$, the only difference is that now we use an integer instead of a variable.
It becomes a little bit complex when dealing with reference, in transition rule for $\LET x=*y \IN e$, we update the register of $x$ to $H(R(y))$,
which means the memory address of the variable $y$.
If you want to $\MKREF$, first it should satisfies that the variable $y$ doesn't used in memory before, after executing, heap
turns to $H(y\ra v)$, here $v$ represents a new memory cell.
We divide the expression $\IFZERO x\THEN e_1 \ELSE e_2$ into two parts, $e_1$ will be given if $x\ra 0$ in the initial register,
otherwise we will get $e_2$.
When handling with $\ALIAS$ and $\ASSERT$, we introduce $\bf{AliasFail}$ and $\bf{AssertFail}$ to the rules. They will be thrown if the memory cell or
the environment can't hold for some states.
The expression about function definition, we replace it by the function body itself and we also need to remember to bind $x$ to
the result of $[\tilde{y}/\tilde{x}]e'$ and then calculate $e$, which is written by $\LET x=[] \IN e$.

\section{TODO}

\begin{itemize}
\item Running example.
\item Inference (constraint generation) algorithm.
\end{itemize}

\newcommand\DOM{\textit{dom}}
\newcommand\EXTEND[1]{\{{#1}\}}

%edit by shiki
\begin{figure}[t]
  \leavevmode
  %% \infrule[]{
  %% }{
  %%   \langle H, R, E[n] \rangle \ra_D \langle H, R(n), E\rangle
  %% }
  \infrule[]{
  }{
    \langle H, R, \overrightarrow{E}, E[x] \rangle \ra_D \langle H, R, \overrightarrow{E}, E[R(x)]\rangle
  }
  \infrule[]{
    x' \notin \DOM(R)
  }{
    \langle H, R, \overrightarrow{E}, E[\LET x=y \IN e] \rangle \ra_D \langle H, R \EXTEND{x' \mapsto R(y)}, \overrightarrow{E}, E[[x'/x]e]\rangle
  }
  \infrule[]{
    x' \notin \DOM(R)
  }{
    \langle H, R, \overrightarrow{E}, E[\LET x=n \IN e] \rangle \ra_D \langle H, R \EXTEND{x' \mapsto n}, \overrightarrow{E}, E[[x'/x]e]\rangle
  }
  \infrule[]{
    R(x) = 0
  }{
    \langle H, R, \overrightarrow{E}, E[\IFZERO x\THEN e_1 \ELSE e_2] \rangle \ra_D \langle H, R, \overrightarrow{E}, E[e_1]\rangle
  }
  \infrule[]{
    R(x) \neq 0
  }{
    \langle H, R, \overrightarrow{E}, E[\IFZERO x\THEN e_1 \ELSE e_2] \rangle \ra_D \langle H, R, \overrightarrow{E}, E[e_2]\rangle
  }
  \infrule[]{
    a \notin \DOM(H) \andalso x' \notin \DOM(R)
  }{
    \langle H, R, \overrightarrow{E}, E[\LET x=\MKREF y \IN e] \rangle \ra_D \langle H \EXTEND{a \mapsto v}, R \EXTEND{x' \mapsto y}, \overrightarrow{E}, E[[x'/x]e]\rangle
  }
  \infrule[]{
    R(y) = a \andalso H(a) = v \andalso x' \notin \DOM(R)
  }{
    \langle H, R, \overrightarrow{E}, E[\LET x=*y \IN e] \rangle \ra_D \langle H, R\EXTEND{x' \mapsto v}, \overrightarrow{E}, E[[x'/x]e]\rangle
  }
  \infrule[]{f^l(\tilde{x})=e\in P
  }{
    \langle H, R, \overrightarrow{E}, E[\LET x=f^l(\tilde{y}) \IN e] \rangle \ra_D \langle H, R, \overrightarrow{E}:E[\LET x=[] \IN e], E[[\tilde{y}/\tilde{x}]e']\rangle
  }
  \infrule[]{
  }{
    \langle H, R, \overrightarrow{E}, E[x:=y; e] \rangle \ra_D \langle H, R(x'\ra y), \overrightarrow{E}, E[[x'/x]e]\rangle
  }
  \infrule[]{H, R\models (e_1=e_2)
  }{
    \langle H, R, \overrightarrow{E}, E[\ALIAS(e_1=e_2); e_3] \rangle \ra_D \langle H, R, \overrightarrow{E}, E[e_3]\rangle
  }
  \infrule[]{H, R\not\models (e_1=e_2)
  }{
    \langle H, R, \overrightarrow{E}, E[\ALIAS(e_1=e_2); e_3] \rangle \ra_D \bf{AliasFail}
  }
  \infrule[]{H, R\models \phi
  }{
    \langle H, R, \overrightarrow{E}, E[\ASSERT(\phi)] \rangle \ra_D \langle H, R, \overrightarrow{E}, E\rangle
  }
  \infrule[]{H, R\not\models \phi
  }{
    \langle H, R, \overrightarrow{E}, E[\ASSERT(\phi)] \rangle \ra_D \bf{AssertFail}
  }
\caption{Transition Rules.}
\label{fig:transitionRules}
\end{figure}







\bibliographystyle{plain}
\bibliography{main.bib}

\end{document}

%%% Local Variables:
%%% mode: latex
%%% TeX-master: t
%%% End:
